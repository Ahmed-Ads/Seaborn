\documentclass{report}
\usepackage[utf8]{inputenc}
\usepackage{graphicx}
\usepackage{ragged2e} 
\usepackage{fancyhdr} 
\usepackage{pgfplots}
\usepackage{amsmath}
\usepackage{subcaption}
\usepackage[left=4cm, right=4cm, top=3cm, bottom=3cm]{geometry}
\usepackage[colorlinks=true, urlcolor=blue]{hyperref}
\usepackage{xcolor}
\usepackage{listings}
\usepackage{float}
\usepackage{datetime}



% Set up code listing style
\lstset{
    language=Python,
    basicstyle=\ttfamily\small,
    keywordstyle=\color{blue},
    commentstyle=\color{green!40!black},
    stringstyle=\color{red},
    showstringspaces=false,
    breaklines=true,
    frame=single,
    backgroundcolor=\color{gray!5},
    tabsize=4,
    literate={*}{{\char42}}1
             {-}{{\char45}}1
}


% Set up fancy header
\pagestyle{fancy}
\fancyhf{} 
\fancyhead[C]{Ahmed Mohamed Ads\\} 
\fancyhead[L]{Faculty of Engineering}
\fancyhead[R]{Mansoura University} 

\title{Seaborn Summary}
\author{Ahmed Mohamed Ads} 
\date{\formatdate{29}{4}{2025} \\
1 Dhu al-Qi‘dah 1446 AH
}

\parindent0pt
\begin{document}
\maketitle

% Title Section
\section*{\huge Seaborn}

\subsection*{seaborn} 
A Python library used for enhancing the style of different graphs.

\subsubsection*{Seaborn updates matplotlib's rc parameters to improve aesthetics}

\begin{lstlisting}[language=Python]
import seaborn as sns
sns.set() 
plt.plot(x, y)
sns.set_style("white")
\end{lstlisting}

\begin{itemize}
\item \lstinline|import seaborn as sns|: importing the seaborn library to use it with the name sns.
\item \lstinline|sns.set()|: function for setting the seaborn default style on the plot by updating the matplotlib's rc parameters, and see \href{https://seaborn.pydata.org/generated/seaborn.set_theme.html#seaborn.set_theme}{Seaborn set documentation} for more details.
\item \lstinline|plt.plot()|: plotting the data to watch the difference.
\item \lstinline|sns.set_style("white")|: for making a specific background style, here a plain white background.
\end{itemize}


\begin{figure}[H]
\centering
\caption{Diffrences between plots}
\begin{subfigure}[b]{0.45\textwidth}
\includegraphics[width=1\textwidth]{"C:/LaTex/Seaborn/Seaborn\_photos/figure1.PNG"}
\caption{Original plot}
\end{subfigure}
\hfill
\centering
\begin{subfigure}[b]{0.45\textwidth}
\includegraphics[width=1\textwidth]{"C:/LaTex/Seaborn/Seaborn\_photos/figure2.PNG"}
\caption{After using sns.set()}
\end{subfigure}
\begin{subfigure}[b]{0.45\textwidth}
\includegraphics[width=1\textwidth]{"C:/LaTex/Seaborn/Seaborn\_photos/figure3.PNG"}
\caption{After using sns.set\_style()}
\end{subfigure}
\end{figure}

\subsection*{Versions}

\begin{lstlisting}[language=Python]
import matplotlib        #importing matplotlib library
matplotlib.__version__   #Getting the current version of matplotlib
import seaborn as sns    #importing seaborn library as sns
sns.__version__          #Getting the current version of seaborn
\end{lstlisting}

\subsection*{Seaborn with pandas} 

\subsubsection*{Let Seaborn group, aggregate, and plot your pandas dataframes}


\begin{lstlisting}[language=Python]
cars = sns.load_dataset('mpg')
type(cars)
cars.dropna(inplace=True)
cars.shape
cars.head()
sns.relplot(x='model_year', y='mpg', col='origin', hue='cylinders', data=cars)
\end{lstlisting}

\begin{itemize}
\item \lstinline|cars = sns.load\_dataset('mpg')|: loading the built-in MPG dataset that will be plotted.
\item \lstinline|type(cars)|: getting the type of the object cars output : \texttt{pandas.core.frame.DataFrame}
\item \lstinline|cars.shape|: plotting the data to watch the difference.
\item \lstinline|cars.dropna(inplace=True)|: removing all the rows in the dataset that contain missing values (NaN).
\item \lstinline|cars.head()|: getting the first five rows in the data set.
\item \lstinline|sns.relplot(x='model_year', y='mpg', col='origin', hue='cylinders', data=cars)|: for creating a plot (scatter plot) and for the full signature visit \href{https://seaborn.pydata.org/generated/seaborn.relplot.html}{Seaborn replot documentation}. 
\end{itemize}


\begin{figure}[H]
\centering
\caption{Fuel efficiency (mpg) over model year, separated by car origin plot}
\includegraphics[width=1\textwidth]{"C:/LaTex/Seaborn/Seaborn\_photos/figure4.PNG"}
\end{figure}

\subsection*{Seaborn: KDEplot} 

\subsubsection*{KDE (Kernel Density Estimation)}
\begin{lstlisting}[language=Python]
sns.kdeplot(data_to_be_plotted, fill=True, bw_adjust= value, cumulative=True)
\end{lstlisting}

\begin{itemize}
\item \lstinline|sns.kdeplot(data_to_be_plotted, fill=True, bw_adjust=value, cumulative=True|:  
  \begin{itemize}
  \item \texttt{sns.kdeplot()}  for plotting KDE plot.
  \item \texttt{fill} for filling the area under the curve.
  \item \texttt{bw\_adjust} for changing the bandwidth of the plot.
  \item \texttt{cumulative = True} for making the plot a cumulative distribution function instead of a probability density function.
  \end{itemize}
  \texttt{Note:} there are some changes in the parameters of this function. Visit \href{https://seaborn.pydata.org/generated/seaborn.kdeplot.html}{Seaborn KDE plot documentation} to see it and for more. 
\end{itemize}



\begin{figure}[H]
\centering
\caption{KDE Plots}
\begin{subfigure}[b]{0.45\textwidth}
\includegraphics[width=1\textwidth]{"C:/LaTex/Seaborn/Seaborn\_photos/figure5.PNG"}
\caption{default KDE plot}
\end{subfigure}
\hfill
\centering
\begin{subfigure}[b]{0.45\textwidth}
\includegraphics[width=1\textwidth]{"C:/LaTex/Seaborn/Seaborn\_photos/figure6.PNG"}
\caption{Filled KDE plot}
\end{subfigure}
\begin{subfigure}[b]{0.45\textwidth}
\includegraphics[width=1\textwidth]{"C:/LaTex/Seaborn/Seaborn\_photos/figure7.PNG"}
\caption{KDE plot with a different bandwidth}
\end{subfigure}
\begin{subfigure}[b]{0.45\textwidth}
\includegraphics[width=1\textwidth]{"C:/LaTex/Seaborn/Seaborn\_photos/figure8.PNG"}
\caption{KDE plot as a cumulative distribution function}
\end{subfigure}

\end{figure}

\subsection*{Bivariate KDEplot}

\begin{lstlisting}[language=Python]
sns.kdeplot(x=data_1, y=data_2,levels=value, fill=True, cbar=True)
\end{lstlisting}

\begin{itemize}
\item \lstinline|sns.kdeplot(x=data_1, y=data_2,levels=value, fill=True, cbar=True)|:
  \begin{itemize}
  \item \texttt{sns.kdeplot(x=data\_1, y=data\_2)}  for plotting bivariate KDE plot.
  \item \texttt{levels} for changing the number of levels in the plot.
  \item \texttt{fill} for filling the plot.
  \item \texttt{cbar = True} for adding a color bar for the filled plot.
  \end{itemize}
\end{itemize}

\begin{figure}[H]
\centering
\caption{Bivariate KDE Plots}
\begin{subfigure}[b]{0.45\textwidth}
\includegraphics[width=1\textwidth]{"C:/LaTex/Seaborn/Seaborn\_photos/figure9.PNG"}
\caption{Default bivariate KDE plot}
\end{subfigure}
\hfill
\centering
\begin{subfigure}[b]{0.45\textwidth}
\includegraphics[width=1\textwidth]{"C:/LaTex/Seaborn/Seaborn\_photos/figure10.PNG"}
\caption{Bivariate KDE plot with changing levels}
\end{subfigure}
\begin{subfigure}[b]{0.45\textwidth}
\includegraphics[width=1\textwidth]{"C:/LaTex/Seaborn/Seaborn\_photos/figure11.PNG"}
\caption{Filled bivariate KDE plot}
\end{subfigure}
\begin{subfigure}[b]{0.45\textwidth}
\includegraphics[width=1\textwidth]{"C:/LaTex/Seaborn/Seaborn\_photos/figure12.PNG"}
\caption{Bivariate KDE plot with color bar}
\end{subfigure}
\end{figure}

\subsection*{Hist plot}

\begin{lstlisting}[language=Python]
sns.histplot(penguins.bill_length_mm)
sns.histplot(x='bill_length_mm', data=penguins)
sns.histplot(y='bill_length_mm', data=penguins)
sns.histplot(x='bill_length_mm', data=penguins, kde=True)
sns.histplot(x='bill_length_mm', data=penguins, bins=20)
sns.histplot(x='bill_length_mm', data=penguins, binwidth=10)
sns.histplot(x='bill_length_mm', data=penguins, binrange=(30, 60))
sns.histplot(x='bill_length_mm', data=penguins, stat='density')
sns.histplot(x='bill_length_mm', data=penguins, stat='probability', cumulative=True)
sns.histplot(x='bill_length_mm', data=penguins, hue='species')
sns.histplot(x='bill_length_mm', data=penguins, hue='species', element = 'poly')
\end{lstlisting}


\begin{itemize}
    \item \texttt{For more about Hist plot  visit  \href{https://seaborn.pydata.org/generated/seaborn.histplot.html}{Seaborn Hist plot documentation}}
\end{itemize}

\begin{itemize}
  \item \lstinline|sns.histplot(penguins.bill_length_mm) sns.histplot(x='bill_length_mm', data=penguins)| 
  \item \lstinline| sns.histplot(x='bill_length_mm', data=penguins)| : Both for plotting a histogram and visit \href{https://seaborn.pydata.org/generated/seaborn.histplot.html}{Seaborn Hist plot documentation} for more.
  \item \lstinline| sns.histplot(y='bill_length_mm', data=penguins)| : for plotting a horizontal histogram.
  \item \lstinline|sns.histplot(x='bill_length_mm', data=penguins,kde=True, bins=value, binwidth=value, binrange=(value_1, value_2))|:
  \begin{itemize}
    \item \texttt{kde=True} for adding a KDE plot.
    \item \texttt{bins = value} dividing the histogram into a desired number of bins.
    \item \texttt{binwidth = value} choosing a specific bandwidth for the bin in the plot.
    \item \texttt{binrange = (value1, value2)} dividing the histogram bins with a specific range.
    \item \texttt{stat ='density', stat = 'probability'} for making the values on y a density representation or a probability representation, and see the documentation at the top for more.
    \item \texttt{cumulative = True} represent a cumulative function.
    \item \texttt{hue = ' '} splits the data sent by the hue column and colors each different species by a color.
   \item \texttt{element = ' '} Visual representation of the histogram statistic.
  \end{itemize}
\end{itemize}




\begin{figure}[H]
  \centering
  \caption{Histogram Plots – Part 1}
  \begin{subfigure}[b]{0.45\textwidth}
    \includegraphics[width=1\textwidth]{"C:/LaTex/Seaborn/Seaborn\_photos/figure13.PNG"}
    \caption{Default Histogram plot}
  \end{subfigure}
  \hfill
  \begin{subfigure}[b]{0.45\textwidth}
    \includegraphics[width=1\textwidth]{"C:/LaTex/Seaborn/Seaborn\_photos/figure14.PNG"}
    \caption{Horizontal Histogram plot}
  \end{subfigure}
  \begin{subfigure}[b]{0.45\textwidth}
    \includegraphics[width=1\textwidth]{"C:/LaTex/Seaborn/Seaborn\_photos/figure15.PNG"}
    \caption{Histogram plot with KDE plot}
  \end{subfigure}
  \hfill
  \begin{subfigure}[b]{0.45\textwidth}
    \includegraphics[width=1\textwidth]{"C:/LaTex/Seaborn/Seaborn\_photos/figure16.PNG"}
    \caption{Histogram plot with 20 bins}
  \end{subfigure}
  \begin{subfigure}[b]{0.45\textwidth}
    \includegraphics[width=1\textwidth]{"C:/LaTex/Seaborn/Seaborn\_photos/figure17.PNG"}
    \caption{Histogram plot with a specific bandwidth (10)}
  \end{subfigure}
  \hfill
  \begin{subfigure}[b]{0.45\textwidth}
    \includegraphics[width=1\textwidth]{"C:/LaTex/Seaborn/Seaborn\_photos/figure18.PNG"}
    \caption{Histogram plot with a specific binrange (30, 60)}
  \end{subfigure}
\end{figure}

\begin{figure}[H]
  \centering
  \caption{Histogram Plots – Part 2}
  \begin{subfigure}[b]{0.45\textwidth}
    \includegraphics[width=1\textwidth]{"C:/LaTex/Seaborn/Seaborn\_photos/figure19.PNG"}
    \caption{Histogram plot with a density representation}
  \end{subfigure}
  \hfill
  \begin{subfigure}[b]{0.45\textwidth}
    \includegraphics[width=1\textwidth]{"C:/LaTex/Seaborn/Seaborn\_photos/figure20.PNG"}
    \caption{Histogram plot with a probability representation}
  \end{subfigure}
  \begin{subfigure}[b]{0.45\textwidth}
    \includegraphics[width=1\textwidth]{"C:/LaTex/Seaborn/Seaborn\_photos/figure21.PNG"}
    \caption{Histogram plot with a probability cumulative representation}
  \end{subfigure}
  \hfill
  \begin{subfigure}[b]{0.45\textwidth}
    \includegraphics[width=1\textwidth]{"C:/LaTex/Seaborn/Seaborn\_photos/figure22.PNG"}
    \caption{Histogram plot with the hue}
  \end{subfigure}
  \begin{subfigure}[b]{0.45\textwidth}
    \includegraphics[width=1\textwidth]{"C:/LaTex/Seaborn/Seaborn\_photos/figure23.PNG"}
    \caption{Histogram visual representation using poly}
  \end{subfigure}
\end{figure}







\subsection*{ECDF plot} 
\subsubsection*{ECDF (Empirical Cumulative Distribution Function)}

\begin{itemize}
    \item \texttt{For more about ECDF plot  visit \href{https://seaborn.pydata.org/generated/seaborn.ecdfplot.html}{Seaborn ECDF plot documentation}}
  \end{itemize}

\begin{itemize}
  \item \texttt{\textcolor{green!70!black}{Advantage}}
  \begin{itemize}
    \item \texttt{No binning or smoothing}
    \item \texttt{Compare Category distribution is easy} 
  \end{itemize}
\item \texttt{\textcolor{red!70!black}{Dis Advantage}}
 \begin{itemize}
    \item \texttt{Hard to see Central Tendecies}
    \item \texttt{Hard to detect as we have a Bimodal distribution} 
  \end{itemize}
\end{itemize}


\begin{lstlisting}[language=Python]
# For plotting an ECDF plot
sns.ecdfplot(x='tip', data=tips); 

#Using hue to split the plot
sns.ecdfplot(x='tip', data=tips, hue = 'day');

#Using state for changing the y representation.
sns.ecdfplot(x='tip', data=tips,stat = 'count', hue = 'day');

#Using lw for styling (changing line width)
sns.ecdfplot(x='tip', data =tips,hue='day', lw = 3);
\end{lstlisting}


\begin{figure}[H]
\centering
\caption{ECDF plots}
  \begin{subfigure}[b]{0.45\textwidth}
    \includegraphics[width=1\textwidth]{"C:/LaTex/Seaborn/Seaborn\_photos/figure24.PNG"}
    \caption{Default ECDF plot}
  \end{subfigure}
  \begin{subfigure}[b]{0.45\textwidth}
    \includegraphics[width=1\textwidth]{"C:/LaTex/Seaborn/Seaborn\_photos/figure25.PNG"}
    \caption{ECDF plot using hue}
  \end{subfigure}
 \begin{subfigure}[b]{0.45\textwidth}
    \includegraphics[width=1\textwidth]{"C:/LaTex/Seaborn/Seaborn\_photos/figure26.PNG"}
    \caption{ECDF plot using stat and hue}
  \end{subfigure}
\begin{subfigure}[b]{0.45\textwidth}
    \includegraphics[width=1\textwidth]{"C:/LaTex/Seaborn/Seaborn\_photos/figure27.PNG"}
    \caption{ECDF plot with a styled line width}
  \end{subfigure}
\end{figure}



\subsection*{Box plot} 
\begin{itemize}
    \item \texttt{Seaborn classifies the box plot as a categorical distribution plot}
    \item \texttt{For more about Box plot  visit \href{https://seaborn.pydata.org/generated/seaborn.boxplot.html}{Seaborn Box plot documentation}} 
    \item \texttt{\texttt{Note:} you can find some styling that isn't mentioned in the Seaborn documentation, but it is in the Matplotlib documentation, so visit \href{https://matplotlib.org/stable/index.html}{Matplotlib documentation}}
  \end{itemize}


\begin{lstlisting}[language=Python]
# For plotting a  Box plot
# (for a horizontal Box plot reverse the x and y and the reverse is true)
sns.boxplot(x= cars.mpg) 
sns.boxplot(x= cars.origin,y = cars.mpg)
sns.boxplot(x= 'origin' ,y='mpg', data = cars)
sns.boxplot(y='origin', x='mpg', data=cars)


# For plotting a vertical Box plot 
sns.boxplot(cars.mpg)        
sns.boxplot(y=cars.mpg)    


# Using hue to split the plot
sns.boxplot(x= 'origin' ,y='mpg', hue='cylinder', data = cars)


# Using some styling
sns.boxplot(x='mpg', y='origin', data=cars,hue='newer_model', linewidth=2.5, fliersize=5)

\end{lstlisting}


\begin{figure}[H]
\centering
\caption{Box plots}
  \begin{subfigure}[b]{0.45\textwidth}
    \includegraphics[width=1\textwidth]{"C:/LaTex/Seaborn/Seaborn\_photos/figure28.PNG"}
    \caption{Default Box plot}
  \end{subfigure}
  \begin{subfigure}[b]{0.45\textwidth}
    \includegraphics[width=1\textwidth]{"C:/LaTex/Seaborn/Seaborn\_photos/figure29.PNG"}
    \caption{Vertical Box plot with x and y data}
  \end{subfigure}
\begin{subfigure}[b]{0.45\textwidth}
    \includegraphics[width=1\textwidth]{"C:/LaTex/Seaborn/Seaborn\_photos/figure30.PNG"}
    \caption{Horizontal Box plot with x and y data}
  \end{subfigure}
 \begin{subfigure}[b]{0.45\textwidth}
    \includegraphics[width=1\textwidth]{"C:/LaTex/Seaborn/Seaborn\_photos/figure31.PNG"}
    \caption{Box plot using hue to split data}
  \end{subfigure}
\begin{subfigure}[b]{0.45\textwidth}
    \includegraphics[width=1\textwidth]{"C:/LaTex/Seaborn/Seaborn\_photos/figure32.PNG"}
    \caption{Box plot with a bigger fliersize and linewidth}
  \end{subfigure}
\end{figure}









\subsection*{Violin plot} 
\begin{itemize}
    \item \texttt{Seaborn classifies the box plot as a categorical distribution plot}
    \item \texttt{For more about Box plot  visit \href{https://seaborn.pydata.org/generated/seaborn.boxplot.html}{Seaborn Box plot documentation}} 
    \item \texttt{\texttt{Note:} you can find some styling that isn't mentioned in the Seaborn documentation, but it is in the Matplotlib documentation, so visit \href{https://matplotlib.org/stable/index.html}{Matplotlib documentation}}
  \end{itemize}


\begin{lstlisting}[language=Python]
# For plotting a  Violin plot
sns.violinplot(cars.displacement)
sns.violinplot(x=cars.cylinders, y=cars.displacement);
sns.violinplot(x='cylinders', y='displacement',data=cars)


# For plotting a Horizontal Violin plot 
sns.violinplot(x = cars.displacement)
 

# Using hue to split the plot
sns.violinplot(x='cylinders', y='displacement', hue='origin', data=cars);


# Using split to split the violin plot into parts for each category in the x variable.
sns.violinplot(x='cylinders', y='displacement', hue='origin', data=cars[cars.origin.isin(['japan', 'europe'])], split=True)

# Using some styling
sns.boxplot(x='mpg', y='origin', data=cars,hue='newer_model', linewidth=2.5, fliersize=5)


#changing the inner part of the violin plot
sns.violinplot(x='cylinders', y='displacement', hue='origin', data=cars[cars.origin.isin(['japan', 'europe'])], split=True, inner='quartiles')


# Using some styling like changing the bandwidth
sns.violinplot(x=cars.cylinders, y=cars.displacement, bw_method =0.2);

\end{lstlisting}


\begin{figure}[H]
\centering
\caption{Violin plots}
  \begin{subfigure}[b]{0.45\textwidth}
    \includegraphics[width=1\textwidth]{"C:/LaTex/Seaborn/Seaborn\_photos/figure33.PNG"}
    \caption{Default Violin plot}
  \end{subfigure}

  \begin{subfigure}[b]{0.45\textwidth}
    \includegraphics[width=1\textwidth]{"C:/LaTex/Seaborn/Seaborn\_photos/figure34.PNG"}
    \caption{Horizontal Violin plot}
  \end{subfigure}
  \begin{subfigure}[b]{0.45\textwidth}
    \includegraphics[width=1\textwidth]{"C:/LaTex/Seaborn/Seaborn\_photos/figure35.PNG"}
    \caption{Vertical Violin plot with x and y data}
  \end{subfigure}
 \begin{subfigure}[b]{0.45\textwidth}
    \includegraphics[width=1\textwidth]{"C:/LaTex/Seaborn/Seaborn\_photos/figure36.PNG"}
    \caption{Violin plot using hue to split data}
  \end{subfigure}
\begin{subfigure}[b]{0.45\textwidth}
    \includegraphics[width=1\textwidth]{"C:/LaTex/Seaborn/Seaborn\_photos/figure37.PNG"}
    \caption{Violin plot using split}
  \end{subfigure}
\begin{subfigure}[b]{0.45\textwidth}
    \includegraphics[width=1\textwidth]{"C:/LaTex/Seaborn/Seaborn\_photos/figure38.PNG"}
    \caption{Violin plot with changed inner part}
  \end{subfigure}
\begin{subfigure}[b]{0.45\textwidth}
    \includegraphics[width=1\textwidth]{"C:/LaTex/Seaborn/Seaborn\_photos/figure39.PNG"}
    \caption{Violin plot with changed bandwidth}
  \end{subfigure}
\end{figure}



\subsection*{Swarm plot} 
\begin{itemize}
    \item \texttt{In Swarm plot, there is no overlap}
    \item \texttt{For more about Swarm plot  visit \href{https://seaborn.pydata.org/generated/seaborn.swarmplot.html}{Seaborn Swarm plot documentation}} 
  \end{itemize}


\begin{lstlisting}[language=Python]
# For plotting a  Swarm plot
# (For a horizontal Box plot reverse the x and y and the reverse is true)
swarmplot(cars.horsepower);
sns.swarmplot(x=cars.origin, y=cars.horsepower)
sns.swarmplot(x='origin', y='horsepower', data=cars)

# For plotting a Horizontal Swarm plot 
sns.swarmplot(x=cars.horsepower)
 

# Using hue to split the Swarm plot
sns.swarmplot(x='origin', y='horsepower', hue='cylinders', data=cars)


# Using  dodge to  split the Swarm plot data horizontally
sns.swarmplot(x='origin', y='horsepower', hue='cylinders', data=cars, dodge=True)

# Changing Swarm plot marker
sns.swarmplot(cars.horsepower, marker= 'X')


# Plotting a swam plot and a Box plot 
sns.boxplot(x=usa.cylinders, y=usa.horsepower)
sns.swarmplot(x=usa.cylinders, y=usa.horsepower)


# Plotting a Swarm plot and a Violin plot 
sns.violinplot(x=usa.cylinders, y=usa.horsepower)
sns.swarmplot(x=usa.cylinders, y=usa.horsepower)

\end{lstlisting}


\begin{figure}[H]
\centering
\caption{Swarm plots – Part 1}
  \begin{subfigure}[b]{0.45\textwidth}
    \includegraphics[width=1\textwidth]{"C:/LaTex/Seaborn/Seaborn\_photos/figure40.PNG"}
    \caption{Default Swarm plot}
  \end{subfigure}
  \begin{subfigure}[b]{0.45\textwidth}
    \includegraphics[width=1\textwidth]{"C:/LaTex/Seaborn/Seaborn\_photos/figure41.PNG"}
    \caption{Horizontal Swarm plot}
  \end{subfigure}
  \begin{subfigure}[b]{0.45\textwidth}
    \includegraphics[width=1\textwidth]{"C:/LaTex/Seaborn/Seaborn\_photos/figure42.PNG"}
    \caption{Vertical Swarm plot with x and y data}
  \end{subfigure}
 \begin{subfigure}[b]{0.45\textwidth}
    \includegraphics[width=1\textwidth]{"C:/LaTex/Seaborn/Seaborn\_photos/figure43.PNG"}
    \caption{Swarm plot using hue to split data}
  \end{subfigure}
\end{figure}
\begin{figure}[H]
\centering
\caption{Swarm plots  – Part 2}

\begin{subfigure}[b]{0.45\textwidth}
    \includegraphics[width=1\textwidth]{"C:/LaTex/Seaborn/Seaborn\_photos/figure44.PNG"}
    \caption{Swarm plot using dodge to split data horizontally}
  \end{subfigure}
\begin{subfigure}[b]{0.45\textwidth}
    \includegraphics[width=1\textwidth]{"C:/LaTex/Seaborn/Seaborn\_photos/figure45.PNG"}
    \caption{Swarm plot with Box plot}
  \end{subfigure}
\begin{subfigure}[b]{0.45\textwidth}
    \includegraphics[width=1\textwidth]{"C:/LaTex/Seaborn/Seaborn\_photos/figure46.PNG"}
    \caption{Swarm plot with Violin plot}
  \end{subfigure}
\begin{subfigure}[b]{0.45\textwidth}
    \includegraphics[width=1\textwidth]{"C:/LaTex/Seaborn/Seaborn\_photos/figure47.PNG"}
    \caption{Swarm plot with an X marker}
  \end{subfigure}
\end{figure}




\subsection*{Strip plot} 
\begin{itemize}
    \item \texttt{For more about Strip plot  visit \href{https://seaborn.pydata.org/generated/seaborn.stripplot.html}{Seaborn Strip plot documentation}} 
  \end{itemize}


\begin{lstlisting}[language=Python]
# For plotting a  Strip plot
sns.stripplot(cars.weight)
sns.stripplot(x=cars.weight, y=cars.origin)
sns.stripplot(x='weight',y='origin', data=cars)


# For plotting a Horizontal Strip plot 
sns.stripplot(x=cars.weight)
 

# Using hue to split the Strip plot
sns.stripplot(x ='weight', y='origin', hue='cylinders', data=cars)


# Using  dodge to  split the Strip plot data horizontally
sns.stripplot(x ='weight', y='origin', hue='cylinders', data=cars, dodge=True)


# Changing spread using jitter
sns.stripplot(x=cars.weight, y=cars.origin,data=cars, hue='cylinders', jitter=.25);


# Changing markers and size of markers
sns.stripplot(x=cars.weight, y=cars.origin,data=cars, hue='cylinders', marker='*', size=7)   


# Changing the transparency of the markers 
sns.stripplot(x=cars.weight, y=cars.origin,data=cars, hue='cylinders', size =7, alpha=0.25)


\end{lstlisting}


\begin{figure}[H]
\centering
\caption{Strip plots – Part 1}
  \begin{subfigure}[b]{0.45\textwidth}
    \includegraphics[width=1\textwidth]{"C:/LaTex/Seaborn/Seaborn\_photos/figure48.PNG"}
    \caption{Default Strip plot}
  \end{subfigure}
  \begin{subfigure}[b]{0.45\textwidth}
    \includegraphics[width=1\textwidth]{"C:/LaTex/Seaborn/Seaborn\_photos/figure49.PNG"}
    \caption{Horizontal Strip plot}
  \end{subfigure}
  \begin{subfigure}[b]{0.45\textwidth}
    \includegraphics[width=1\textwidth]{"C:/LaTex/Seaborn/Seaborn\_photos/figure50.PNG"}
    \caption{Strip plot with x and y data}
  \end{subfigure}
 \begin{subfigure}[b]{0.45\textwidth}
    \includegraphics[width=1\textwidth]{"C:/LaTex/Seaborn/Seaborn\_photos/figure51.PNG"}
    \caption{Strip plot using hue to split data}
  \end{subfigure}
\end{figure}
\begin{figure}[H]
\centering
\caption{Strip plots – Part 2}
\begin{subfigure}[b]{0.45\textwidth}
    \includegraphics[width=1\textwidth]{"C:/LaTex/Seaborn/Seaborn\_photos/figure52.PNG"}
    \caption{Strip plot using dodge to split data horizontally}
  \end{subfigure}
\begin{subfigure}[b]{0.45\textwidth}
    \includegraphics[width=1\textwidth]{"C:/LaTex/Seaborn/Seaborn\_photos/figure53.PNG"}
    \caption{Strip plot using jitter}
  \end{subfigure}
\begin{subfigure}[b]{0.45\textwidth}
    \includegraphics[width=1\textwidth]{"C:/LaTex/Seaborn/Seaborn\_photos/figure54.PNG"}
    \caption{Strip plot with different markers and marker sizes}
  \end{subfigure}
\begin{subfigure}[b]{0.45\textwidth}
    \includegraphics[width=1\textwidth]{"C:/LaTex/Seaborn/Seaborn\_photos/figure55.PNG"}
    \caption{Strip plot with alpha}
  \end{subfigure}
\end{figure}










\subsection*{Scatter plot} 

\begin{itemize}
   \item \texttt{Scatter plot is considered a rational plot}
    \item \texttt{For more about Scatter plot  visit \href{https://seaborn.pydata.org/generated/seaborn.scatterplot.html}{Seaborn Scatter plot documentation}} 
  \end{itemize}

\begin{lstlisting}[language=Python]
# For plotting a  Scatter plot
sns.scatterplot(x=diamonds.carat, y=diamonds.price)
sns.scatterplot(x='carat', y='price', data=diamonds)


# Using hue to split the Scatter plot
sns.scatterplot(x='carat', y='price', hue='cut', data=diamonds)
sns.scatterplot(x='carat', y='price', hue='depth', data=diamonds)

# Using  palette to  pass a specific color to the markers
sns.scatterplot(x='carat', y='price', hue='cut', data=diamonds, palette=['purple', '#55CCCC'])


# Using sizes to split the data  using marker size
sns.scatterplot(x='carat', y='price', size='cut', data=diamonds, sizes=[150, 50])
sns.scatterplot(x='carat', y='price', size='cut', data=diamonds, sizes=[150, 50])


# Using style and sizes to change markers and the size of markers
sns.scatterplot(x='carat', y='price', hue='cut', style='color', size='depth', data=diamonds)


\end{lstlisting}


\begin{figure}[H]
\centering
\caption{Scatter plots}
  \begin{subfigure}[b]{0.45\textwidth}
    \includegraphics[width=1\textwidth]{"C:/LaTex/Seaborn/Seaborn\_photos/figure56.PNG"}
    \caption{Default Scatter plot}
  \end{subfigure}
  \begin{subfigure}[b]{0.45\textwidth}
    \includegraphics[width=1\textwidth]{"C:/LaTex/Seaborn/Seaborn\_photos/figure57.PNG"}
    \caption{Scatter plot using hue(cut) to Split data}
  \end{subfigure}
  \begin{subfigure}[b]{0.45\textwidth}
    \includegraphics[width=1\textwidth]{"C:/LaTex/Seaborn/Seaborn\_photos/figure58.PNG"}
    \caption{Scatter plot using hue(depth) to Split data}
  \end{subfigure}
  \begin{subfigure}[b]{0.45\textwidth}
    \includegraphics[width=1\textwidth]{"C:/LaTex/Seaborn/Seaborn\_photos/figure59.PNG"}
    \caption{Scatter plot with colors we made}
  \end{subfigure}
  \begin{subfigure}[b]{0.45\textwidth}
    \includegraphics[width=1\textwidth]{"C:/LaTex/Seaborn/Seaborn\_photos/figure60.PNG"}
    \caption{Scatter plot using sizes to Split data by marker size}
  \end{subfigure}
  \begin{subfigure}[b]{0.45\textwidth}
    \includegraphics[width=1\textwidth]{"C:/LaTex/Seaborn/Seaborn\_photos/figure61.PNG"}
    \caption{Scatter plot with splitted data by marker and size}
  \end{subfigure}
\end{figure}







\subsection*{Line plot} 
\begin{itemize}
    \item \texttt{For more about Line plot  visit \href{https://seaborn.pydata.org/generated/seaborn.lineplot.html}{Seaborn Line plot documentation}} 
  \end{itemize}

\begin{lstlisting}[language=Python]
# For plotting a  Line plot
sns.lineplot(x=park.Day, y=park.Occupancy)
sns.lineplot(x='Day', y='Occupancy', data=park)


# Using n_boot to control the number of bootstrap  resamples to estimate uncertainty
sns.lineplot(x='Day', y='Occupancy', data=park, n_boot=1000)


# Using  ci to  change the confidence interval level for the line plot 
sns.lineplot(x='Day', y='Occupancy', data=park, errorbar=('ci',70))


# Using an estimator changes the aggregate function for the data we want to visualize, without modifying the plot itself
sns.lineplot(x='Day', y='Occupancy', data=park, estimator='std')


# Using hue and style to split and change the style of the data
sns.lineplot(x='Day', y='Occupancy', data=park, hue="Location", style='Location')



\end{lstlisting}

\begin{itemize}
  \item {\textbf{Note:}} To turn off the bootstrapped confidence intervals, set \texttt{ci=None} to trigger early exit within Seaborn code. A conditional checks for this case and completely bypasses the bootstrapping procedure if \texttt{ci} is set to \texttt{None}. This saves time if confidence intervals are not needed!
\end{itemize}

\begin{figure}[H]
\centering
\caption{Line plots}
  \begin{subfigure}[b]{0.45\textwidth}
    \includegraphics[width=1\textwidth]{"C:/LaTex/Seaborn/Seaborn\_photos/figure62.PNG"}
    \caption{Default Line plot}
  \end{subfigure}
  \begin{subfigure}[b]{0.45\textwidth}
    \includegraphics[width=1\textwidth]{"C:/LaTex/Seaborn/Seaborn\_photos/figure63.PNG"}
    \caption{Line plot with changing the number of bootstrap}
  \end{subfigure}
  \begin{subfigure}[b]{0.45\textwidth}
    \includegraphics[width=1\textwidth]{"C:/LaTex/Seaborn/Seaborn\_photos/figure64.PNG"}
    \caption{Line plot using ci to change confidence interval}
  \end{subfigure}
  \begin{subfigure}[b]{0.45\textwidth}
    \includegraphics[width=1\textwidth]{"C:/LaTex/Seaborn/Seaborn\_photos/figure65.PNG"}
    \caption{Line plot using estimator to make the aggregate function(std)}
  \end{subfigure}
  \begin{subfigure}[b]{0.45\textwidth}
    \includegraphics[width=1\textwidth]{"C:/LaTex/Seaborn/Seaborn\_photos/figure66.PNG"}
    \caption{Line plot using hue and style to Split and style data}
  \end{subfigure}
\end{figure}










\subsection*{Regression plot} 
\begin{itemize}
    \item \texttt{For more about Line plot  visit \href{https://seaborn.pydata.org/generated/seaborn.regplot.html}{Seaborn Regression plot documentation}} 
  \end{itemize}

\begin{lstlisting}[language=Python]
# For plotting a  Reg plot
sns.regplot(x=diamonds.carat, y=diamonds.price)
sns.lineplotsns.regplot(x='carat', y='price', data=diamonds)



# Using fit_reg to remove the line and confidence interval
sns.regplot(x='carat', y='price', data=diamonds, fit_reg=False)


# Using  scatter to  remove scatter points 
sns.regplot(x='carat', y='price', data=diamonds, scatter=False)


# Using  ci to  change the confidence interval level for the line plot  from (None, 100)
sns.regplot(x='carat', y='price', data=diamonds,ci = 70)


# Changing spread using  x_jitter
sns.regplot(x='cut_value', y='price', data=diamonds, x_jitter=.02)


#Using x_estimator for applying an aggregation function to the x-values before fitting the regression line
sns.regplot(x='cut_value', y='price', data=diamonds, x_estimator=np.mean)


#Using order to change the order of values fit a polynomial regression instead of a simple straight line (linear regression)
sns.regplot(x='carat', y='price', data=diamonds, order=2)

#Styling, for example, changing the line style, color, and width
sns.regplot(x='carat', y='price', data=diamonds, ci=None, line_kws={'lw': 4, 'color': 'black', 'linestyle': '-.'})


\end{lstlisting}

\begin{itemize}
  \item \textbf{Note:} The \texttt{regplot} function allows customization using various parameters:
  \begin{itemize}
    \item \texttt{order}: Controls the polynomial degree for fitting non-linear regression curves. For example, \texttt{order=2} fits a quadratic regression curve.
    \begin{lstlisting}
    sns.regplot(x='carat', y='price', data=diamonds, order=2)
    \end{lstlisting}
    \item \texttt{robust}: Applies robust regression, making the model more resistant to outliers, thus reducing the influence of extreme values on the fit.
    \begin{lstlisting}
    sns.regplot(x='carat', y='price', data=diamonds, robust=True)
    \end{lstlisting}
    \item \texttt{logistic}: Fits a logistic regression (sigmoid curve), useful for binary outcome variables.
    \begin{lstlisting}
    sns.regplot(x='carat', y='price', data=diamonds, logistic=True)
    \end{lstlisting}
    \item \texttt{lowess}: Applies locally weighted scatterplot smoothing (LOESS), fitting a smooth curve to the data without assuming a global functional form.
    \begin{lstlisting}
    sns.regplot(x='carat', y='price', data=diamonds, lowess=True)
    \end{lstlisting}
  \end{itemize}
\end{itemize}



\begin{figure}[H]
\centering
\caption{Regression plots – Part 1}
  \begin{subfigure}[b]{0.45\textwidth}
    \includegraphics[width=1\textwidth]{"C:/LaTex/Seaborn/Seaborn\_photos/figure67.PNG"}
    \caption{Default Reg plot}
  \end{subfigure}
  \begin{subfigure}[b]{0.45\textwidth}
    \includegraphics[width=1\textwidth]{"C:/LaTex/Seaborn/Seaborn\_photos/figure68.PNG"}
    \caption{Reg plot without the line and the ci}
  \end{subfigure}
  \begin{subfigure}[b]{0.45\textwidth}
    \includegraphics[width=1\textwidth]{"C:/LaTex/Seaborn/Seaborn\_photos/figure69.PNG"}
    \caption{Reg plot without the scatter points}
  \end{subfigure}
  \begin{subfigure}[b]{0.45\textwidth}
    \includegraphics[width=1\textwidth]{"C:/LaTex/Seaborn/Seaborn\_photos/figure70.PNG"}
    \caption{Reg plot without the changed ci}
  \end{subfigure}
  \begin{subfigure}[b]{0.45\textwidth}
    \includegraphics[width=1\textwidth]{"C:/LaTex/Seaborn/Seaborn\_photos/figure71.PNG"}
    \caption{Reg plot using x\_jitter to chage spread}
  \end{subfigure}
\end{figure}


\begin{figure}[H]
\centering
\caption{Regression plots – Part 2}
  \begin{subfigure}[b]{0.45\textwidth}
    \includegraphics[width=1\textwidth]{"C:/LaTex/Seaborn/Seaborn\_photos/figure72.PNG"}
    \caption{Reg plot using x\_estimator for appling mean }
  \end{subfigure}
  \begin{subfigure}[b]{0.45\textwidth}
    \includegraphics[width=1\textwidth]{"C:/LaTex/Seaborn/Seaborn\_photos/figure73.PNG"}
    \caption{Reg plot with a polynomial regression}
  \end{subfigure}
  \begin{subfigure}[b]{0.45\textwidth}
    \includegraphics[width=1\textwidth]{"C:/LaTex/Seaborn/Seaborn\_photos/figure74.PNG"}
    \caption{Reg plot with changed line}
  \end{subfigure}
\end{figure}









\subsection*{Conclusion}
At the end, this was a fast summary for the \texttt{Seaborn Library} by me,\\ and if it was useful, don't forget to pray for me \texttt{and for all our brothers in Palestine and Sudan}. And thank you :).

\center{\textcopyright 2025 Ahmed\_Ads \\}

\noindent\hrulefill % Adds a horizontal line
\vspace{1em} % Adds vertical space

\end{document}

